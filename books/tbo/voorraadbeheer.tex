\documentclass[../../samenvatting.tex]{subfiles}
\begin{document}

\section{Voorraadbeheer}
\subsection{Belang van voorraden}

Voorraden zijn noodzakelijk voor de financiële cyclus, om toegevoegde waarde te kunnen maken. 

Voorraden zijn nodig om de productiecyclus soepel te laten verlopen. Met behulp van overschotten kan de planning vlotter verlopen, worden productieverliezen opgevangen, kan er een buffer worden opgezet voor bepaalde machines en worden instelkosten verminderd.

Voorraden zijn ook belangrijk bij de marketing- en verkoopafdeling. Door een voldoende voorraad wordt de verkoop losgekoppeld van productie, kunnen schommelingen in verkoop opgevangen worden, wordt de levertermijn verkort en andere problemen verborgen.

De omvang van een voorraad wordt gemeten met de \emph{rotatie}. Dit geeft aan hoeveel keer de voorraad van een bepaald product gedurende een bepaalde tijd doorheen de omzet loopt.

Het aanhouden van een grotere vooraad betekent extra kosten

\end{document}