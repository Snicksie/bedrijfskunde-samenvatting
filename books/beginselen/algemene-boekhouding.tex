\documentclass[../../samenvatting.tex]{subfiles}
\begin{document}

\section{Algemene boekhouding}

\subsection{Inleiding}

De onderneming heeft zowel \emph{externe middelen} (grondstoffen, verbruiksgoederen, diensten in onderaanneming) als \emph{interne middelen} (arbeid, reclame, gebouwen, rollend materieel, machines).

De algemene boekhouding is een weerspiegeling van de inkomsten en uitgaven van een onderneming, de aanwendingen en bronnen, opbrengsten en kosten.

De industriële boekhouding is bedoeld om de kostprijs te kunnen analyseren en eventueel te verlagen.

\subsection{Principes van de algemene boekhouding}
De algemene boekhouding probeert zo werkelijksheidgetrouw als mogelijk de waarden en verandering van waarden binnen de onderneming weer te geven, van zowel aanwendingen(bezittingen en vorderingen) als bronnen(verplichtingen en kapitaal).

De zaaktheorie maakt een onderscheid tussen het vermogen van de onderneming en de eigenaar. De eigenaar of aandeelhouder is een buitenstaander in de boekhouding.

\begin{equate}
Bezittingen/vorderingen = Schulden_{Eigenaar} + Schulden_{Derden}
\end{equate}

Iedere verrichting gaat gepaard met een dubbele inschrijving in de boekhouding. Het gaat ofwel om een \emph{innerlijke waardeverschuiving} (enkel verandering binnen $B$, $S_E$ of $S_D$), een \emph{waarde-omzetting} met effect op $B$ en $S_D$ (aankopen verhogen $B$ en $S_D$, betaling van schulden verlaagt $B$ en $S_D$), of een \emph{waarde-omzetting} met effect op $B$ en $S_E$ of $S_D$ en $S_E$ (winst of verlies).

\subsection{De balans}
De balans wordt in twee delen verdeeld: links (\emph{actief}: gebruik van bronnen / bezittingen) en rechts (\emph{passief}: bronnen/schulden van de onderneming). Een onderneming kan niet meer bronnen aanwenden dan beschikbaar.

De activa worden gerangschikt volgens liquiditeitsgraad (snelheid waarin dit in geld kan worden omgezet, van laag naar hoog). Deze worden verdeeld in vaste activa(worden binnen het jaar niet omgezet: gebouwen, machines) en courante activa(worden binnen het jaar omgezet: exploitatiewaarden(grondstoffen), realiseerbare waarden(klanten), beschikbaar(bank, kas)).

De passiva worden gerangschikt volgens eisbaarheidsgraad. Deze worden verdeeld in bestendige kapitalen(eigen middelen en schulden op lange en middellange termijn) en passief op korte termijn(schulden op korte termijn).

Deze voorstelling zorgt dat je de financiële positie van een bedrijf kan bepalen. Je moet deze zowel verticaal als horizontaal lezen en bepalen of een onderneming op een bepaald moment financieel gezond is.

Balansen zijn bedoeld als een waarheidsgetrouwe manier voor bedrijfsleiders en andere geïnteresseerden om de financiële status van het bedrijf te bekijken. Deze moet voldoen aan drie factoren:
\begin{itemize}
    \item \emph{Balanswaarheid}: een waarheidsgetrouw beeld geven
    \item \emph{Balanseenheid}: een éénvormige gedragslijn volgen, zodat balansen vergeleken kunnen worden
    \item \emph{balansklaarheid}: een eenduidige presentatie van de balans
\end{itemize}

Er bestaan voor de boekhouding drie verschillende soorten ondernemingen, die van klein naar groot telkens meer boekhouding moeten doen:
\begin{itemize}
    \item kleine ondernemingen
    \item kmo's
    \item grote ondernemingen
\end{itemize}

Afschrijvingen zijn permanente waardeverminderingen aan lange-termijn materieel, zodat de kosten hiervan over verschillende jaren worden gespreid. Waardeverminderingen zijn tijdelijk en zijn afhankelijk van fluctuerende prijzen.

Overlopende rekeningen zijn bedoeld om kosten/opbrengsten in de juiste periode te klasseren.

Als bepaald materieel voor een te laag bedrag op de balans staat, kan er een supplement worden toegevoegd, in combinatie met herwaarderingsmeerwaarden.

De gevolgde gedragslijn moet opgeschreven worden, zodat men deze stelselmatig kan controleren.

Sommige bedrijven verspreiden ook een sociale balans, waarin cijfers staan betreffende het personeel. Dit is geen vastgelegd element.

\subsection{De rekening en het journaal}

De rekening vormt een chronologische weergave van de vermogensbestanddelen. Toename's in de rekening worden links gezet (debetzijde), afnames staan aan de creditzijde (rechts). Er worden meerdere rekeningen gemaakt, waarbij sommige rekeningen elementen van de balans zijn. Voor elke verrichting worden twee rekeningen gehanteerd, die vermeld worden in het journaal.

De jaarrekening bestaat uit:
\begin{itemize}
    \item De \emph{balans} die een overzicht geeft van het vermogen.
    \item De \emph{resultatenrekening} die het resultaat toelicht.
    \item De \emph{toelichting} die bepaalde zaken verder in detail uitlegt.
\end{itemize}

\subsection{De resultatenrekening}
De resultatenrekening geeft weer hoe het vermogen van de onderneming werd gebruikt en of er winst of verlies is gemaakt. Deze bestaat grofweg uit twee delen: bedrijfskosten en -opbrengsten en overige kosten en opbrengsten, die verder opgedeeld kunnen worden in financiële kosten/opbrengsten en uitzonderlijke kosten/opbrengsten.

De resultatenrekening geeft informatie over het rendement van een bedrijfsactiviteit.

De resultatenrekening moet informatie geven over de omvang en samenstellende bestanddelen van het product en de inkomsten die de realisatie heeft opgebracht. Ook wordt er vermeld welk deel van winst/verlies aan eigen kapitaal wordt toegevoegd/verwijderd en welk deel aan kapitaalverschaffers(eigenaar) wordt uitgekeerd/gevraagd.

In de resultatenrekening worden kosten naar hun aard ingedeeld. 

Bij het opstellen van de resultatenrekening moet rekening gehouden worden met inflatie en fluctuerende prijzen, aangezien deze de gegevens negatief kunnen beïnvloeden.
\end{document}