\documentclass[../../samenvatting.tex]{subfiles}
\begin{document}

\section{Economisch model}
\subsection{Inleiding}
Operationeel onderzoek helpt om te bepalen welke producten uit het assortiment bijdragen tot bedrijfswinst. Dit maakt gebruik van het economisch model.

\subsection{Economisch model}
\begin{equate}
    Max\ bedrijfsResultaat = \sum_{i=1}^{n} q_i \cdot (v_i - k_i) - vaste\ Kosten\\
    q_i = hoeveelheid\ product\ i \\
    v_i = verkoopprijs\ product\ i \\
    k_i = evenredige\ kosten\ product\ i \\
    n = aantal\ producten
\end{equate}

Het bedrijfsresultaat wordt beperkt door de technische beperkingen van capaciteit en commerciële beperkingen (maximaal voorziene verkoop van product i).

In operationeel onderzoek worden de beperkingen met het economisch model tot een Lineair Programma gevormd, waar een verzameling $q_i$ uitkomt zodat het bedrijfsresultaat optimaal is. Aangezien een dergelijk probleem vaak omvangrijk is, wordt dit maar één keer in de zoveel tijd berekend.

\subsection{Schaduwkost}
De marginale toename/afname van het optimale bedrijfsresultaat bij verandering van capaciteit wordt \emph{schaduwkost} genoemd. Als er één extra arbeider bijkomt en de onderneming op maximum capaciteit is, zal de ondernemingswinst toenemen. Schaduwkost wordt gebruikt om te berekenen of een bepaalde verandering winst zal opleveren.

\subsection{Transferprijs}
De transferprijs wordt gebruikt om de prestatie van afdelingen die een halfproduct aanleveren te kunnen beoordelen. De transferprijs is de som van alle meerkosten en schaduwkosten van een bepaalde afdeling.

De transferprijs kan ook gebruikt worden om te bepalen of een product zelf gemaakt moet worden of moet uitbesteed worden (toeleveren of uitbesteden / make or buy), hoeveel marketingkosten er gemaakt kunnen worden en onderhoudskosten (hoeveel geld mag men uitgeven om 1 uur panne te besparen?).

\subsection{Knelpuntcalculatie}
Ieder bedrijf heeft een stel beperkingen, waarvan er een aantal dwingerder zijn en dus knelpunten/\emph{bottlenecks} zijn. Deze knelpunten zijn bijvoorbeeld machine-uren, arbeidsuren, zeldzame grondstoffen, ... Met de knelpuntcalculatie kan bepaald worden welk product de grootste marge per uur van de knelpuntcapaciteit heeft. Dit product is het meest rendabele product.

Indien er geen rekening gehouden wordt met knelpuntcapaciteit, wordt de marge vergroot door structureel niet-rendabele producten te verkopen die boven de evenredige kost liggen of door alle bestellingen met positieve marge bij laagconjunctuur aan te nemen. 
\end{document}