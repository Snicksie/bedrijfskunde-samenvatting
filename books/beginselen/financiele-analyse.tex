\documentclass[../../samenvatting.tex]{subfiles}
\begin{document}

\section{Financiële analyse en financiering}
\subsection{Inleiding}

Financiële analyse geeft inzicht in de financiële toestand van een onderneming. Er zijn verschillende technieken:
\begin{itemize}
    \item Horizontale/Tijdanalyse: Vergelijken van opeenvolgende balansen/resultatenrekeningen om de evolutie binnen de tijd op te volgen.
    \item Verticale/structuuranalyse: Diverse posten van de balans/resultatenrekening procentueel uitdrukken om de structuur te analyseren.
    \item Vermogensstroom- en cashflowanalyse: Welke bronnen trekt de onderneming geld uit aan en waarvoor wordt dit gebruikt, om de financierings- en investeringspolitiek te analyseren.
    \item Ratio-analyse: verhoudingen van getallen analyseren.
\end{itemize}

\subsection{Vermogensstroomanalyse}

Vermogen stroomt op een continue manier door een onderneming. Sommige ondernemingen voegen een vermogenssstroomtabel bij hun jaarrekening. Deze tabel bevat een kolom met bronnen en een kolom met aanwendingen. Hierin kan gekozen worden om verschillende elementen te benadrukken:
\begin{itemize}
    \item Een indeling tussen interne en externe bronnen/aanwendingen.
    \item Een indeling waarin aan de ene kant vermogensaantrekking en desinvestering, aan de andere kant vermogensafstoting en investering staat.
    \item Verticale indeling volgens de verschillende activiteiten van een bedrijf.
\end{itemize}

\subsection{Ratio-analyse}
Bedrijfsgoed moet gefinancieerd worden met financieringsmiddelen van minstens gelijke duur, om financieel in evenwicht te zijn.

\begin{equate}
bedrijfskapitaal = permanente\ middelen - vaste\ activa = \\
vlottende\ activa - schulden\ op\ korte\ termijn
\end{equate}

\begin{equate}
    current\ ratio = \frac{vlottende\ activa}{vreemd\ vermogen\ op\ korte\ termijn}
\end{equate}

De current ratio bepaalt de relatieve belangrijkheid van het bedrijfskapitaal. Deze ratio moet hoger zijn dan 1, een goede liquiditeit ligt tussen $1.2$ en $1.8$.

\begin{equate}
    voorraadrotatie = \frac{omzet\ (aan\ kostprijs)}{gemiddelde\ voorraad\ (per\ jaarperiode)}
\end{equate}

De voorraadrotatie geeft aan hoevaak de voorraad wordt omgezet per jaar en geeft dus de opslagduur aan.

\begin{equate}
    klantenrotatie = \frac{omzet\ (aan\ kostprijs)}{gemiddeld\ klantenkrediet}
\end{equate}

\begin{equate}
    leveranciersrotatie = \frac{omzet\ (aan\ kostprijs)}{gemiddeld\ leverancierskrediet}
\end{equate}

Voor een goede liquiditeit zal een bedrijf een hoge klantenrotatie hebben (snel betalen) en een lagere leveranciersrotatie (later betalen). Deze geven aan in welke mate de onderneming schulden kan vereffenen bij normale bedrijfsactiviteit (going concern).

\begin{equate}
    solvabiliteitsratio = \frac{eigen\ vermogen}{vreemd\ vermogen}
\end{equate}

Een hogere solvabiliteitsratio betekent dat een bedrijf meer in waarde mag verminderen, voor vereffening van schulden onmogelijk wordt. De financiële afhankelijkheid wordt uitgedrukt door de schuldenratio.

\begin{equate}
    schuldenratio = \frac{vreemd\ vermogen}{totaal\ vermogen}
\end{equate}


De toegevoegde waarde geeft de verhoging in waarde aan van het product:
\begin{equate}
    Toegevoegde\ Waarde = Bedrijfsopbrengsten - Intermediair\ gebruik
\end{equate}

Deze toegevoegde waarde bestaat uit de personeelskosten, fiscale kosten, dividenden en afschrijvingen.

Cashflow is een maatstaf om te bepalen in hoeverre de onderneming activiteiten kan financieren met interne financiële middelen.

\begin{equate}
Winst\ na\ belasting + Afschrijvingen + Uitzonderlijke\ Afschrijvingen\ +\\
Afschrijving\ kosten\ uitgifte\ leningen - kapitaalsubsidies + voorzieningen\ voor\\\ risico's\ en\ kosten + voorzieningen\ voor\ uitzonderlijke\ kosten = CASHFLOW\\
Ofwel: Cashflow = Nettowinst + Afschrijvingen
\end{equate}

\begin{equate}
    Rentabiliteit\ Totaal\ Vermogen = \frac{winst\ van\ het\ boekjaar + financi\text{ë}le\ kosten}{eigen\ vermogen + vreemd\ vermogen}\\
    = \frac{winst + financi\text{ë}le\ kosten}{omzet} \cdot \frac{omzet}{totaal\ vermogen}\\
    winstmarge \cdot omloopsnelheid\ vermogen
\end{equate}
\begin{equate}
Rentabiliteit\ Eigen\ Vermogen = \frac{winst\ van\ het\ boekjaar}{eigen\ vermogen}\\
= Rentabiliteit\ Totaal\ Vermogen \cdot\ financi\text{ë}le\ hefboom
\end{equate}

In bepaalde gevallen kan het gunstig zijn om vreemd vermogen te gebruiken. Zolang de interesten op vreemd vermogen kleiner zijn dan de winst die de onderneming hiermee kan verdienen, is het gunstig. Dit kan echter wel een verslechterde solvabiliteit geven en meer contractuele verplichtingen.

\subsection{Financiering}
Een onderneming heeft eigen vermogen (wat niet terugbetaald moet worden) en vreemd vermogen (wat op termijn terugbetaald moet worden, schulden).

Aandelen zijn onderdeel van het eigen vermogen, de aandeelhouder is mede-eigenaar van het bedrijf. Winst wordt uitbetaald aan de aandeelhouders als dividenden. Ook kunnen bonusaandelen worden uitgedeeld.

Kapitaalverhoging kan gebeuren door nieuwe aandeelhouders(aandelenemissie) aan te trekken of oude aandeelhouders meer geld te laten investeren (zelffinanciering).

Het verschil tussen de nominale waarde van een aandeel en de uitgifteprijs wordt \emph{agio} of \emph{uitgiftepremie} genoemd.

\emph{Obligaties} zijn langetermijnschulden met vaste vergoedings- en aflossingsverplichtingen. 

Een \emph{converteerbare obligatielening} is een obligatie die om te zetten is in aandeling, indien gewenst. 

Een \emph{hypothecaire lening} is een langdurige lening tegen onderpand van terreinen of gebouwen. 

Bij een \emph{leasingovereenkomst} huurt een \emph{lessee} een kapitaalgoed van de \emph{lessor}, met een bepaalde vastgelegde duur. Onderhoud is ten koste van de huurder. Op die manier hoeft een huurder niet direct de volle pot te betalen.

\emph{Leverancierskrediet} wordt veroorzaakt door de mogelijkheid van korte-termijn betalingen bij leveranciers, in plaats van directe betaling.

\emph{Korte bankkredieten} is een krediet bij een bank voor een vast bedrag, voor een vastgestelde termijn of onbepaalde duur.

\emph{Rekening-courantkrediet} heeft een krediet-limiet en moet niet volledig opgenomen worden.

\emph{Factoring} is het principe waarin een \emph{factor} de vordering van een bedrijf overkoopt en daarbij ook de risico's overkoopt.
\end{document}