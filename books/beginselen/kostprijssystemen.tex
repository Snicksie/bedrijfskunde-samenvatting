\documentclass[../../samenvatting.tex]{subfiles}
\begin{document}

\section{Kostprijssystemen}
\subsection{Inleiding}
\emph{Structuurbeslissingen} zijn beslissingen op lange termijn, die vooral betrekking hebben op vaste kosten (investeringen, reorganisatie, samenwerkingen met andere bedrijven, nieuwe producten).

\emph{Beheersbeslissingen} zijn dagelijkse besluiten en vooral belangrijk wegens de hoeveelheid beslissingen. Hiervoor is vaak snel beschikbare en vernieuwbare informatie nodig.

\emph{Industriële boekhouding} is bedoeld om de bedrijfsleiding te informeren over:
\begin{itemize}
    \item De kost van ieder product
    \item Het kostenbeheer van iedere afdeling
    \item De rendabiliteit van iedere verkoop
\end{itemize}

\subsection{Kostprijselementen}
Kosten kunnen verdeeld worden in verschillende categorieën (grondstoffen, duurzame productiemiddelen, menselijke arbeid, ...). Ook kunnen ze opgesplitst worden op basis van afhankelijk van bedrijfsdrukte (vaste/variabele kosten) en directe toewijsbaarheid aan producten (direct/indirecte kosten).

\subsection{Vaste en variabele kosten}
Vaste kosten zijn binnen bepaalde grenzen onafhankelijk van bedrijfsdrukte, maar kunnen wel stapsgewijs verhoogd zijn bij verhoogde bedrijfsdrukte.

Variabele kosten zijn afhankelijk van de bedrijfsdrukte. Er zijn 3 soorten afhankelijkheden:
\begin{itemize}
    \item proportioneel variabel: veranderen recht evenredig met de bedrijfsdrukte
    \item degressief variabel: veranderen minder dan evenredig. Bijvoorbeeld kostenbesparingen bij grotere bedrijfsdrukte.
    \item progressief variabel: veranderen meer dan evenredig. Bijvoorbeeld inefficiëntie bij capaciteitsgrens bedrijf.
\end{itemize}

\subsection{Directe en indirecte kosten}
Directe kosten zijn direct toe te wijzen aan een specifiek product, bestelling of activiteit. Het gaat hier om loonkosten en materiaalkosten.

Directe materiaalkosten kunnen op verschillende manieren berekend worden:
\begin{itemize}
    \item FIFO: goederen worden verstrekt tegen de originele aankoopprijs. De oudste hoeveelheid wordt eerst opgemaakt.
    \item LIFO: goederen worden verstrekt tegen de originele aankoopprijs. De nieuwste hoeveelheid wordt eerst opgemaakt.
    \item Gemiddelde aankoopprijs: regelmatig wordt een nieuwe gemiddelde prijs berekend, zodat schommelingen worden gedempt.
    \item Vervangingswaarde: goederen worden verstrekt tegen de huidig geldende marktprijs.
    \item Standaardprijs: er wordt een bepaalde prijs vastgesteld voor één jaar.
\end{itemize}

Indirecte kosten kunnen niet direct gekoppeld worden aan één product of bestelling. De toeslagmethode die hiervoor gebruikt wordt is gebaseerd op het feit dat de indirecte kosten veroorzaakt door een product evenredig zijn met de directe kosten.

Indirecte kosten worden met de primitieve toeslagmethode met één criterium verrekend:
\begin{itemize}
    \item Toewijzing op basis van directe loonkosten ($\frac{indirecte\ kosten}{directe\ lonen}$)
    \item Toeslag per eenheid eindproduct
    \item Toeslag op basis van directe materiaalkosten ($\frac{indirecte\ kosten}{directe\ materiaalkosten}$)
    \item Toeslag op basis van directe kosten ($\frac{indirecte\ kosten}{directe\ loonkosten + directe\ materiaalkosten}$)
\end{itemize}

Met de verfijnde toeslagmethode worden de indirecte kosten zo goed mogelijk gesplitst en wordt nagegaan met welk deel van de directe kosten het meest verband houden. Er wordt dus met meerdere verdeelsleutels gewerkt. De toeslagmethode geeft niet noodzakelijk een correct beeld, aangezien deze redelijk arbitrair gekozen is.

Met de kostenplaatsmethode wordt een bedrijf verdeeld in de volgende centra:
\begin{itemize}
    \item hoofdcentra: hebben een rechtstreekse functie in productie/distributie
    \item hulpcentra: oefenen een activiteit uit om één of meer hoofdcentra in werking te houden
    \item complementaire centra: oefenen een activiteit uit ten voordele van het geheel van alle andere centra.
\end{itemize}

De indirecte kosten worden met (soms ingewikkelde) verdeelsleutels over de hoofdcentra verdeeld. De kosten van complementaire centra worden direct als verlies gerekend.

\subsection{Afschrijvingen}
Voor afschrijvingen zijn verschillende definities:
\begin{itemize}
    \item Geleidelijke verrekening van de investeringskost van duurzame productiemiddelen.
    \item Compensatie voor de waardevermindering van duurzame productiemiddelen.
    \item Desinvestering of recuperatie van de vastgelegde sommen.
\end{itemize}

Afschrijvingen zijn dus geen werkelijke uitgave, maar een vertraagde verrekening van een uitgave die al eerder gedaan is. Deze beïnvloeden de balanswinst, dus meer afschrijven is beter voor de belastingen. Er is een maximumgrens vastgelegd voor het jaarlijks belastingvrij afschrijvingsbedrag, om misbruik te vermijden.

Voor een afschrijving moeten de volgende elementen in rekening worden genomen:
\begin{itemize}
    \item Investeringskost: hierbij moet ook rekening gehouden worden met andere kosten, zoals administratie- en installatiekosten.
    \item Residuwaarde: de waarde na gebruiksduur is meestal klein en wordt vaak verwaarloosd.
    \item Afschrijvingsperiode: de fysische en economische gebruiksduur van materiaal is beperkt, daarna wordt het vervangen en zou de afschrijving vervolledigd moeten zijn.
\end{itemize}

De fiscus heeft bepaalde vastgelegde percentages voor lineaire afschrijving voorgeschreven. Andere methodes zijn degressieve afschrijving door vast percentage op de boekwaarde (double declining balance) totdat dit kleiner is dan de lineaire afschrijving, degressieve afschrijving door sum-of-years-digits methode ($\frac{n-i+1}{\sum_{i=1}^{n}i}$), vertraagde afschrijving (stijgend bedrag per jaar) of afschrijving in verhouding tot prestatie (vrachtwagen per kilometer aanrekenen).

\subsection{Kostprijsberekeningsmethoden}

Er zijn verschillende mogelijkheden om de kostprijs te berekenen, gebaseerd op historische waarden of voorziene waarden(standaardkostprijs):
\begin{itemize}
    \item Klassieke totale kostprijs (full cost, absorptiekostprijs): alle kosten, inclusief de verkoopskosten
    \item Industriële kostprijs: alle kosten met uitzondering van verkoop- en algemene kosten
    \item Evenredige kostprijs (direct costing, marginale kostprijs): enkel kosten die evenredig zijn met de hoeveelheid van ieder product.
\end{itemize}

\subsection{Historische totale kostprijs}
De historische totale kostprijs vertegenwoordigt het totaal van historische kosten voor de fabricage en verkoop van een product. Het probleem is dat de prijzen kunnen veranderen, waardoor deze kostprijs niet veel vooruitzicht geeft. Deze methode geeft een vertekend beeld, omdat er geen mogelijkheid is om efficiëntie te bepalen en geen soepele verkoopspolitiek is(de vaste kosten worden meegeteld in de kostprijs, terwijl die kosten er ook zijn zonder productie en het dus beter is om te verkopen met verlies dan niet te verkopen).

\subsection{Industriële standaardkostprijs}
De industriële standaardkostprijs wordt berekend op basis van voorziene prijzen en kostenbudgetten en geeft aan hoeveel het product mag kosten (\emph{ein sollkost}, een objectief/maatstaf), in plaats van een resultaat. Het verschil tussen de werkelijke kost en de standaardkostprijs wordt als winst/verlies beschouwd.

Met deze standaardkostprijs kan efficiëntie gemeten worden als veranderingen niet een gevolg zijn van bedrijfsdrukte.

\subsection{Evenredige standaardkostprijs (marginale kostprijs)}
De evenredige standaardkostprijs bevat enkel de kosten van de productie van één eenheid zelf, zonder vaste kosten. De vaste kosten worden rechtstreeks naar de resultatenrekening afgevoerd. Hoewel dit een correcter beeld geeft per eenheid, zorgt dit voor een beperkt beeld, omdat de vaste kosten niet meer berekend worden (en mogelijk ook vergeten worden bij verkoop).

Het break even point(dode punt) is het punt waarop de vaste kosten niet meer voor verlies zorgen. Dit is het snijpunt tussen de kosten en omzet.

\subsection{Afwijkingsanalyse}

Standaardkosten hebben een afwijking van de werkelijke kosten. Dit kan veroorzaakt worden door prijsafwijkingen, efficiëntie-afwijkingen of bezettingsafwijkingen. Door de standaard te vergelijken met de werkelijke prijzen, kan bepaald worden welk onderdeel aansprakelijk is voor verlies/winst.

\subsection{Activity Based costing}
De traditionele kostprijssystemen geven in bepaalde gevallen (vooral met hoge overheadkosten) verkeerde informatie, die vooral in een multiproductomgeving verkeerde beslissingen ondersteunt.

Activity Based Costing is gebaseerd op het feit dat activiteiten middelen verbruiken en kosten genereren, in plaats van producten. Zo zal het instellen van een machine tijd kosten en maakt de hoeveelheid minder uit. Hierdoor wordt de nauwkeurigheid van de kostprijs verhoogd. ABC gaat er eigenlijk vanuit dat alle kosten variabel zijn.

\subsection{Target Costing}

Target costing is een systeem om kosten te plannen en te reduceren. Bij het ontwikkelen van een product wordt een target cost vastgelegd, waarbij de ontwerpers en ingenieurs dit target moeten bereiken. Deze techniek is ontwikkeld in Japan. De \emph{cost engineers} hebben praktische ervaring opgedaan om de kostbesparende elementen te kunnen identificeren.

\emph{Concurrent (simulataneous) engineering} betekent dat het product- en ontwerpproces tegelijkertijd worden uitgevoerd, zodat het \emph{over the wall} principe wordt vermeden en er minder tijd verspild wordt aan het ontwerpen.

Target Costing kan ook gebruikt worden om een bestaand product zo goedkoop mogelijk op de markt te brengen door \emph{reverse engineering}. Andere technieken die worden gebruikt zijn:
\begin{itemize}
    \item value-engineering (waarde-analyse): waarde te verhogen door toevoeging van extra elementen en eliminatie van nutteloze zaken
    \item target cost-matrix: tabel met elementen, hun functionaliteit en kosten
    \item cost-tables: database met kosteninformatie over de verschillende product- en productievariabelen
\end{itemize}
\end{document}