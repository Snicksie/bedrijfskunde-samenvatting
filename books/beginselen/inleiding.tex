\documentclass[../../samenvatting.tex]{subfiles}
\begin{document}

\section{Inleiding}
\subsection{Beleid en Ondernemerschap}
Volgens Say zijn de 3 productiefactoren: Men-Materials-Money.

Aangezien ondernemingen telkens complexer worden, zijn er nu 4 productiefactoren: Men-Materials-Money-Management. De managers werken in opdracht van de eigenaars. Een manager heeft de volgende taken: voorspellen - organiseren - bevelen - coördineren - controleren.


\subsection{Duurzame ontwikkeling en Duurzaam ondernemen}
Duurzame ontwikkeling is een evenwichtige ontwikkeling op het vlak van economie, het milieu en de sociale cohesie. VOKA definieert dit als volgt: \emph{Duurzaam ondernemen is het nastreven van een ondernemingsdoelstelling die duurzame waardencreatie betracht, door belang te hechten aan elke betrokkene. Het gaat om het evenwichtig afwegen van economische, sociale en ecologische waarden enerzijds en de korte en lange termijn anderzijds}. Er wordt gewerkt vanuit het principe \emph{People, Planet, Profit}.

\subsection{Maatschappelijke Context en finaliteit van de onderneming}
Een onderneming moet rekening houden met de volgende zaken:
\begin{itemize}
    \item De gebruikers (consumenten) verwachten zaken op gebied van productassortiment, kwaliteit en prijs. Men moet dus produceren wat men kan verkopen.
    \item De verwachtingen van de maatschappij wat betreft tewerkstelling, inkomstenverdeling, ecologie, democratie.
    \item De verwachtingen van het individu (ontplooiing, zekerheid, gebruik van kennis, inspraak, humanisering van arbeid, ...)
    \item De evolutie van de markten (internationalisering, innovatie, ...)
\end{itemize}

Een onderneming is dus een \textbf{economisch systeem} (financiering, productie, marketing) en een \textbf{sociaal systeem} (creëren van welvaart).

Het economisch systeem creeërt meerwaarde.

\begin{equate}
MeerWaarde = Inkomsten - Uitgaven = \\
Vergoeding_{Lonen} + Vergoeding_{Gemeenschap} + Vergoeding_{Kapitaal} + AutoFinanciering
\end{equate}

\subsection{Organisatie en beheer}

Een leider van een onderneming heeft leiderschap, intuïtie en charisma nodig, naast technische vaardigheden.

Het management legt haar objectief vast, bepaalt de strategie die benodigd is en kiest de beste tactiek/organisatie hiervoor.

De verschillende stappen in het beslissingsproces:
\begin{itemize}
    \item probleembeschrijving
    \item modelbouw
    \item gegevensverzameling
    \item oplossingsalgoritme
    \item implementatie van de oplossing
\end{itemize}

In de concurrentiestrijd zijn prijs, tijd en kwaliteit de belangrijkste wapens.

\subsection{Innovatie en onderneming}

Innovatie op het gebied van procédés verhoogt de productie en/of verlaagt de kosten.

Innovatie van producten of diensten zorgt voor differentiatie, waardoor gevoeligheid voor concurrentie wordt verminderd.

Innovatie in de arbeidsorganisatie vormt vaak een noodzakelijke voorwaarde om andere innovatie te doen slagen.

De doorlooptijd(periode tussen twee innovaties) en de \emph{time to market} zijn cruciale factoren voor concurrentie.
\end{document}