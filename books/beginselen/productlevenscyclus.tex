\documentclass[../../samenvatting.tex]{subfiles}
\begin{document}

\section{Productlevenscyclus}
\subsection{Begrip 'productlevenscyclus'}
Een \textbf{product} is datgene wat aan een klantengroep wordt aangeboden en verkocht, wat voldoet aan een behoefte bij die groep.

Een nieuw product kan ontstaan ten gevolge van \emph{market pull} en \emph{technology push}.

Een product heeft vijf fases:
\begin{itemize}
    \item \emph{Introductie} van een product tot de markt
    \item \emph{Vroege groei} waarin het verkoopsvolume telkens sneller toeneemt.
    \item \emph{Late groei} waarin het verkoopsvolume telkens trager stijgt.
    \item \emph{Maturiteit} is de periode waarin het verkoopsvolume relatief constant blijft.
    \item \emph{Verval} waarin de verkoop daalt totdat het product verdwijnt van de markt.
\end{itemize}

Voordat een product op de markt wordt gebracht, is er een \emph{embryonale fase} waarin een idee technisch en commercieel ontwikkeld wordt tot een product.

Er zijn verschillende risicofactoren:
\begin{itemize}
    \item \emph{consumentenrisico}: de kans dat het nieuwe product niet aanvaard wordt.
    \item \emph{technologisch risico}: de kans dat het nieuwe product al verouderd is als het op de markt verschijnt.
\end{itemize}

In de introductiefase zijn het vooral avonturiers die het product kopen en bepalen of een product een succes wordt. In die periode kan de producent binnen bepaalde grenzen zelf de lanceerprijs bepalen. Hij kan deze hoog kiezen (\emph{skimming} van de markt, voordat er concurrenten de prijs verlagen) of laag, om een groot marktaandeel te verkrijgen.

In de vroege groeifase worden concurrenten aangetrokken. Omdat de groei telkens sterker toeneemt, kan het verkoopsvolume van een bedrijf toenemen, maar het marktaandeel dalen. De prijs daalt en wordt een competitief wapen.

In de late groeifase kunnen enkel de sterke concurrenten zich nog handhaven. Productdifferentiatie wordt gebruikt om voordeel te verwerven.

In de maturiteit blijven de verkoopsvolumes constant en zijn er trouwe klanten. De heersers van de markt kunnen de prijs bepalen van de markt. In deze periode worden afspraken gemaakt om prijsdalingen te voorkomen. In deze periode kunnen nieuwe mededingers over het algemeen niet meer op de markt komen.

In de vervalperiode dalen de verkoopsvolumes. Dit kan komen door substitutie door nieuwe producten of veroudering van het product door nieuwe wensen van de klanten. Er ontstaat overcapaciteit, waar prijsdalingen op volgen.

\subsection{Bedenkingen - voorbeelden}

Product kan verschillende dingen betekenen. Enkel voor de bovenste 2 categorieën is PLC relevant.
\begin{itemize}
    \item Productcategorie: auto's
    \item Productsubcategorie: sportauto's, vrachtwagens
    \item Merk: Jaguar
    \item Specifiek model: Golf Diesel
\end{itemize}

De omzetcurve van de innovator en die van de industrietak vallen enkel samen in de introductiefase, totdat andere bedrijven de markt betreden. Daarna zal de innovator de markt moeten delen met anderen.

De PLC kan verlengd worden door een verhoogd gebruik bij de huidige verbruikers (promotieacties), door een meer gevarieerd gebruik (nieuwe toepassingen) of het aanspreken van een nieuwe gebruikersgroep (nieuwe regio, leeftijds- of inkomensklasse).

Het foothill verschijnsel houdt in dat er een tijdelijke verzwakking is in de introductiefase van het product, waarin de verbruikers een opinie over het product vormen.

Modeproducten hebben een heel korte PLC, zonder maturiteitsfase.

De levenscyclus van grondstoffen is een afgeleide vraag, deze cyclus is de omhullende van hun eindproducten.

\subsection{Nut van het concept PLC}

Het PLC-concept helpt het bedrijf bij lange en korte termijnplanning op commercieel en strategisch gebied. Voor een bedrijf is het optimaler om producten in verschillende fases te produceren.

\subsection{PLC en Kostprijs}

De kost van de ontwikkeling van een product krijgt een grotere invloed op de rentabiliteit van een onderneming en worden niet meer als algemene kost gezien. Kosten worden ingebouwd in de blauwdruk.
\end{document}