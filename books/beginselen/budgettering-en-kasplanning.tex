\documentclass[../../samenvatting.tex]{subfiles}
\begin{document}

\section{Budgettering en kasplanning}
\subsection{Inleiding}
Een onderneming heeft een plan nodig: regeren is voorzien. Previsioneel beleid is een beleidsstijl waarbij bewust vooruit wordt gedacht. Een \emph{budget/begroting} zijn de cijfermatige doelstellingen van de doelstellingen van een onderneming. Met deze begroting kunnen beleidsbeslissingen gemaakt worden en de financiële structuur van een bedrijf in evenwicht gehouden worden.

\subsection{Budgettering als beheersproces}
Budgettair beheer is de samenvoeging van de methode der homogene centra en de standaardkostprijs. Hiermee kan per centrum een budget berekend worden, voor alle elementen. Budgettair beleid is verbonden met een aantal fundamentele beheersfuncties:
\begin{itemize}
    \item Planning: iedere actie/keuze vereist planning. 
    \item Organisatie: beperkt budget zorgt dat een goede organisatiestructuur nodig is.
    \item Motvatie: geeft zekerheid over de toekomst en zorgt voor coördinatie.
    \item Controle: automatisch controle op basis van afwijkingen.
\end{itemize}

\subsection{Soorten budgetten}
Er zijn verschillende soorten begrotingen: gewone begrotingen (jaarlijks terugkerende activiteiten) en buitengewone begrotingen (kapitaalsverrichtingen).

Budgetten kunnen ingedeeld worden op tijdsduur(korte/lange termijn), aard(activiteits-, resultaats- of vermogensbudgetten), organisatie(functioneel / persoonlijk), flexibiliteit(variabel, vast, gemengd) en soort(deelbudget of hoofdbudget).

\subsection{Functionele indeling van de budgetten}
Verkoopbudgetten zijn gebaseerd op een grondige marktstudie. Deze budgetten kunnen uitgesplitst zijn naar tijd, geografisch gebied, product(groep) of afnemerscategorie.

Het productiebudget wordt afgeleid van het verkoopbudget en het voorraadbudget.

Het geheel van productieplan, verkoopbudget en voorraadpolitiek vormt het hoofdbudget.

Het investeringsbudget is een langetermijnbudget en wordt bepaald door de goedgekeurde projecten.

Kostenbudgetten bevatten kostenprognoses, van bijvoorbeeld de aankoop van grondstoffen, directe lonen en indirecte kosten.

Het kasbudget geeft de waarschijnlijke kasposities weer als resultaat van geplande activiteiten en geeft de mogelijkheid om de daaruit vloeiende gevolgen te controleren.

Alle budgetten samen geven de mogelijkheid om een previsionele resultatenrekening en balans op te stellen.

\subsection{Budgetcontrole}

Budgetcontrole geeft de mogelijkheid om een planmatig beleid te voeren. Hiervoor wordt onder andere afwijkingsanalyse gebruikt.

In de directe kosten is er een direct verband tussen kostensoort en de einduitvoer. De totale kost is de hoeveelheid vermeningvuldigd met de eindprijs. De afwijking kan opgesplitst worden in efficiëntieverschil en prijsverschil.

Indirecte kosten bevatten een vast en variabel gedeelte, die niet eenvoudig te bepalen zijn. De afwijking kan worden opgesplitst in prijsverschil door schommelingen, efficiëntieverschil en bedrijfsdrukte.

\subsection{Kasplanning}
Een onderneming moet haar geldstromen zo beheren dat ze altijd kan betalen en onverwachte uitgaven kan opvangen.

Voorraden en handelsvorderingen zijn een groot deel van de kosten. Door voorraden te reduceren of handelsvorderingen te versnellen, is er een lagere inzet van middelen nodig. Ook leverancierskrediet zorgt hiervoor.

Met behulp van kasplanning kan maand per maand bepalen wat de verwachten inkomsten en uitgaven zijn. Het verschil tussen deze inkomsten en uitgaven geeft de nettokasstroom. Deze nettokasstroom moet in geval van winst rendabel belegd worden en in geval van verlies bijkomend gefinancierd worden.

\subsection{Besluit}
Een goed budgetsysteem leidt niet automatisch tot resultaten, maar is wel belangrijk. Dit is geen vaststaand systeem, maar moet constant herzien worden.

Kasplanning geeft de mogelijkheid financiële problemen op te lossen en te voorkomen.
\end{document}